\documentclass[runningheads]{llncs}
%
\usepackage[T1]{fontenc}
% T1 fonts will be used to generate the final print and online PDFs,
% so please use T1 fonts in your manuscript whenever possible.
% Other font encondings may result in incorrect characters.
%
\usepackage{graphicx}
% Used for displaying a sample figure. If possible, figure files should
% be included in EPS format.
%
% If you use the hyperref package, please uncomment the following two lines
% to display URLs in blue roman font according to Springer's eBook style:
%\usepackage{color}
%\renewcommand\UrlFont{\color{blue}\rmfamily}
%
\begin{document}
%
\title{Metamycelium: Domain-driven distributed reference architecture for big data systems}
%
%\titlerunning{Abbreviated paper title}
% If the paper title is too long for the running head, you can set
% an abbreviated paper title here
%
\author{Pouya Ataei\inst{1}\orcidID{0000-0002-0993-3574} \and
Alan Litchfield\inst{2,3}\orcidID{0000-0002-3876-0940}}
%
\authorrunning{F. Author et al.}
% First names are abbreviated in the running head.
% If there are more than two authors, 'et al.' is used.
%
\institute{Auckland University of Technology, Princeton NJ 08544, USA \and
\email{pouya.ataei@aut.ac.nz}\\
\url{http://www.springer.com/gp/computer-science/lncs} \and
School of Engineering, Computer and Mathematical Sciences\\
\email{\{abc,lncs\}@uni-heidelberg.de}}
%
\maketitle              % typeset the header of the contribution
%
\begin{abstract}
The abstract should briefly summarize the contents of the paper in
150--250 words.

\keywords{First keyword  \and Second keyword \and Another keyword.}
\end{abstract}
%
%
%
\section{Introduction}
The ubiquity of digital devices and proliferation of software applications have augmented users to generate data at an unprecedented rate. In this day and age, almost all aspects of human life is integrated with some sort of software system, that by large, is processing data, and executing the necessary computations. 

According to Internetlivestats.com \cite{internet2019internet} in one second 3,135,050 emails are sent, 1,151 Instagram photos are uploaded, 6,738 Skype calls are made and 147,084 GB of traffic has gone through the internet. That is, in the last minute, 825.04 terabytes have been transferred through the internet. 

The rapid expansion and evolution of data from an structured element that is passively stored to something that is used to support proactive decision making for business competitive advantage, have dawned a new era, the era of big data. The era of big data began when velocity, variety and volume of data overwhelmed traditional systems used to process that data \cite{ataei2021neomycelia}\cite{AtaeiACIS}. 

Big data is the practice of crunching large sets of heterogenous data to discover patterns and insights for business competitive advantage \cite{AtaeiHype}\cite{Huberty}.

Since the inception of the term, ideas have ebbed and flowed along with the rapid advancements of technology, and many strived to harness the power of data. Nevertheless, big data is not a magical wand that can enchant any business processes and many have failed to absorb the complexity of this new field. According to a recent survey by MIT Technology Review insights in partnership with Databricks, only 13\% of organizations excel at delivering on their data strategy. Another survey by NewVantage Partners highlighted that only 24\% of organizations have successfully adopted big data \cite{NewVantageSurvey}. Sigma computing report indicated that 1 in 4 business experts have given up on getting insights they needed because the data analysis took too long \cite{SigmaSurvey}. Along the lines, McKinsey \& Company \cite{mckines} and Gartner \cite{GartnerSury} demonstrated that approximately only 20\% of organizations have fully adopted big data. These statistics unveil the truth that 






%
% ---- Bibliography ----
%

\bibliographystyle{splncs04}
\bibliography{mybibfile}
\end{document}
